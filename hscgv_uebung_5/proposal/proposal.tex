%\documentclass[10pt,twocolumn]{article}
\documentclass[10pt]{article}
\usepackage{cgp}

\usepackage[ngerman]{babel}
\usepackage[utf8]{inputenc}

\usepackage{graphicx}


\title{Aufgabe 5: Wettbewerb}

\author{Dein Name \thanks{deine-adresse@uni-koeln.de}\\
        \scriptsize Universität zu Köln\\
        Wintersemester 2010
}

\begin{document}

\maketitle

\begin{multicols}{2}

\section{Themen}
In dieser Aufgabe werden folgende Themengebiete bearbeitet:
\begin{itemize}
\item Alles, was Dir einfällt.
\item Etwas Originelles.
\item Z.~B.\ einen kleinen Grafik-Algorithmus demonstrieren.
\item Z.~B.\ zeigen, was Deine Grafikkarte mit Shadern kann.
\item Z.~B.\ etwas auf der GPU berechnen.
\item Oder missbrauche Deine Grafikkarte für andere Zwecke.
\item Z.~B.\ ein kleines Computerspiel.
\item \dots
\end{itemize}


\section{Übersicht}

Eine kurze Beschreibung dessen, was Du vor hast.

\section{Aufbau}

Der grobe Aufbau Deines fertigen Progamms.

\section{Arbeitsschritte}
Hier sollte ein grober Plan stehen, in welchen Schritten Du Dein Ziel
erreichen möchtest.
\subsection{Arbeitsschritt 1: Blah}
\subsection{Arbeitsschritt 2: Fasel}

\section{Bewertungsrichtlinien}

Wenn Du die folgenden Bedingungen erfüllst, dann kannst Du die
Höchstpunktzahl von 20 Punkten erreichen:

\begin{description}
\item[2 Punkte]
        Mindestens ein sinnvoller Aufgabenvorschlag wurde eingereicht.
        Er ist einfallsreich, und es wird aufgezeigt, wie man das gesteckte Ziel
        Schritt für Schritt erreicht.
\item[2 Punkte]
        Das Programm folgt einem objektorientierten Entwurf.
        Der Quelltext ist vollständig in objektorientiertem C++ verfasst.
        Der Code kompiliert auf dem Referenzsystem mit dem GNU C++-Compiler
        ohne Fehler und erzeugt mit den Optionen \texttt{-Wall -O2} keine
        unnötigen Warnungen.
        Auch die anderen Programmierrichtlinien wurden befolgt.
\item[2 Punkte]
        Der gesamte Quelltext ist in englischer Sprache kommentiert und
        die Kommentare erklären die algorithmische Struktur des C++-Codes.
        Die Kommentare sind so formatiert, dass sie das Erzeugen einer
        HTML- und \LaTeX-Dokumentation mit \texttt{doxygen} erlauben.
\item[1 Punkte]
        Das fertige Programm ist von einer README-Datei begleitet, die die
        seine Verwendung beschreibt.
\item[? Punkte]
        Deine Kriterien hier -- so dass sich insgesamt 20 Punkte ergeben!
\end{description}



\bibliographystyle{plain}
\nocite{*}
\bibliography{proposal}

\end{multicols}

\end{document}

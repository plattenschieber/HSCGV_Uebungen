%\documentclass[10pt,twocolumn]{article}
\documentclass[10pt]{article}
\usepackage{cgp}

\usepackage[ngerman]{babel}
\usepackage[utf8]{inputenc}
\usepackage{hyperref}
\hypersetup{colorlinks=true}

\usepackage{graphicx}


\title{Aufgabe 5: Wettbewerb}

\author{Jeronim Morina \thanks{jeronim.morina@smail.uni-koeln.de}\\
        \scriptsize Universität zu Köln\\
        Sommersemester 2015
}

\begin{document}

\maketitle

\begin{multicols}{2}

\section{Themen}
In dieser Aufgabe werden folgende Themengebiete bearbeitet:
\begin{itemize}
    \item Parallelisierung von Raytracing
    \item Visualisierung von GPU Daten
    \item einfache Interaktion mit Visualisierungen
\end{itemize}


\section{Übersicht}

Ziel dieser Aufgabe wird es sein, den in Aufgabe 3 implementierten Raytracer zu parallelisieren. Grafikkarten eignen sich hervorragend um auf ihnen viele Berechnungen parallel auszuführen. Dies erlaubt eine inhärente Implementierung des Raytracers, in dem jeder gesandte Primärstrahl mit einem Thread auf der GPU identifiziert werden kann. Nach einer vollständigen Berechnung der Szene kann das Ergebnis auf der Grafikkarte verweilen um die Daten effizient zu visualisieren. Weiterhin sollen einfache Interaktionen mit der Szene möglich sein. 

\section{Aufbau} 

Der Aufbau des Programms entspricht im Wesentlichen dem der Aufgabe 3. Es müssen jedoch an einigen Stellen Anpassungen gemacht werden. Unter Anderem in dem Aufruf der \textit{shade~()} Methode und ihrer Implementierung. Eine allgemeine Einführung in die Programmierbarkeit von Grafikhardware und in CUDA findet man in Aufgabe 4. 


\section{Arbeitsschritte}
Bei der Implementierung einer parallelen Version eines Raytracers gibt es zu der Ausgangssituation aus Aufgabe 3 Anpassungen zu machen. So ist z.B. die algorithmische Struktur an die Grafikhardware anzupassen und eine korrekte Übermittlung der Daten zur und von der Grafikhardware sicherzustellen. Damit die Szene live gerendert und in ihr interagiert werden kann, bedarf es der Einbindung spezieller APIs.

\subsection{Arbeitsschritt 1: GPU-Portierung von Aufgabe 3}
Die Implementierung der \href{https://github.com/plattenschieber/HSCGV_Uebungen/tree/master/hscgv_uebung_3}{Aufgabe 3} dient als Basis für die Portierung. Dabei beschränken wir uns auf das Rendern von Quadriken. 

\subsection{Arbeitsschritt 2: Live Rendering mit OpenGL}
Nach dem ersten Arbeitsschritt haben wir nun eine voll funktionsfähige Version eines parallelisierten Raytracers vor uns liegen. Bislang werden, zum Darstellen der gerenderten Szene, die Daten auf das Host System zurückkopiert um dann in eine Bitmap Datei geschrieben zu werden.
Wir möchten diesen Schritt gerne überspringen und die gerenderte Szene gleich auf dem Bildschirm darstellen. OpenGL kann jedoch nicht von Hause aus mit CUDA kommunizieren. Um OpenGL nutzen zu können, benötigen wir weiterhin die \textit{CUDA/OpenGL interoperability API}. Die gerenderten Pixel werden anschließend mit der oben genannten Methode gezeichnet. Es bietet sich an, die Daten in eine Textur zu packen und auf einen Bildschirmfüllenden Einheitswürfel zu präsentieren. 

\subsection{Arbeitsschritt 3: Interaktion mit der Szene}
Um die (hoffentlich) deutlich höhere Framerate bei der Nutzung von Grafikhardware auch zu nutzen, bietet sich an eine Manipulation der Szene zuzulassen. Eine Möglichkeit wäre es mit Maus oder Tastatur die Kameraposition im Raum zu verändern. Folglich wäre eine bessere Inspektion der in der Szene gerenderten Objekte möglich. 

\subsection{Arbeitsschritt 4: Optionales}
Da die veränderte Kameraposition mit einer größeren Komplexität des gerenderten Bildes einhergehen kann, wäre es interessant zu wissen, wie schnell die Berechnung der aktuellen Szene von Statten geht. Es gibt zwei Maßzahlen die eine Bewertung der Berechnungsgeschwindigkeit zulassen: 
1. FLOPS - dies sind die Anzahl der Fließkommazahlberechnungen (floating point operations) pro Sekunde. 
2. FPS - die entspricht der Anzahl an vollständig dargestellten Bildern pro Sekunde (frames per second).


\section{Bewertungsrichtlinien}

Wenn Du die folgenden Bedingungen erfüllst, dann kannst Du die
Höchstpunktzahl von 20 Punkten erreichen:

\begin{description}
\item[2 Punkte]
        Mindestens ein sinnvoller Aufgabenvorschlag wurde eingereicht.
        Er ist einfallsreich, und es wird aufgezeigt, wie man das gesteckte Ziel
        Schritt für Schritt erreicht.
\item[2 Punkte]
        Das Programm folgt einem objektorientierten Entwurf.
        Der Quelltext ist vollständig in objektorientiertem C++ verfasst.
        Der Code kompiliert auf dem Referenzsystem mit dem GNU C++-Compiler
        ohne Fehler und erzeugt mit den Optionen \texttt{-Wall -O2} keine
        unnötigen Warnungen.
        Auch die anderen Programmierrichtlinien wurden befolgt.
\item[2 Punkte]
        Der gesamte Quelltext ist in englischer Sprache kommentiert und
        die Kommentare erklären die algorithmische Struktur des C++-Codes.
        Die Kommentare sind so formatiert, dass sie das Erzeugen einer
        HTML- und \LaTeX-Dokumentation mit \texttt{doxygen} erlauben. 
\item[3 Punkte] 
        Die GPU-Implementation bewirkt eine Beschleunigung der Simulation.
\item[3 Punkte] 
        Zur Visualisierung mit OpenGL verbleiben die Daten auf der GPU und werden von dort aus gerendert.
\item[1 Punkte]
        Es kann per Tastendruck zwischen CPU/GPU Berechnung gewechselt werden.
\item[3 Punkte] 
        Es können einfache Manipulationen der Kamera mit der Maus ausgeführt werden.
\item[2 Punkte]
        Es wurde ein Headlight in die Szene eingebaut
\item[1 Punkt]
        Zur Visualisierung werden die benötigten FPS angezeigt.
\item[1 Punkte] 
        Das fertige Programm ist von einer README-Datei begleitet, die die
        seine Verwendung beschreibt.

\end{description}



\bibliographystyle{plain}
\nocite{*}
\bibliography{proposal}

\end{multicols}

\end{document}
